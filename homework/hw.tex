\documentclass[a4paper, 12pt, x11names]{article}

\usepackage[T1, T2A]{fontenc}
\usepackage[utf8]{inputenc}
\usepackage[bulgarian]{babel}
\usepackage[normalem]{ulem}

\usepackage{amsmath}

\usepackage{minted}

\usepackage{tikz}
\usetikzlibrary{er}
\tikzset{multi attribute/.style={attribute, double distance =1.5pt}}
\tikzset{every entity/.style={draw=orange, fill=orange!20}}
\tikzset{every attribute/.style={draw=MediumPurple4, fill=MediumPurple1!20}}
\tikzset{every relationship/.style={draw=Chartreuse4, fill=Chartreuse2!30}}
\newcommand{\key}[1]{\underline{#1}}
\newcommand{\univname}{Софиийски университет "Св. Климент Охридски"\\Факултет по математика и информатика}

\setlength{\parindent}{0mm}

\begin{document}
\begin{titlepage}
\begin{center}
    
\vspace*{.06\textheight}
{\scshape\large \univname\par}\vspace{1.5cm}

{\huge \bfseries{Домашна работа}\par}\vspace{0.7cm}
\textsc{\small по}\\[0.6cm]
\textsc{\Large Бази от данни}\\[0.5cm]
\textsc{\normalsize спец. Информатика, 3 курс, летен семестър,}\\[0.5cm]
\textsc{\normalsize учебна година 2018/19}\\[2cm]
     
\begin{minipage}[t]{0.4\textwidth}
\begin{flushleft} \large
{\large \today}\\[1cm]
София
\end{flushleft}
\end{minipage}
\begin{minipage}[t]{0.4\textwidth}
\begin{flushright} \large
\emph{Изготвил:}\\[0.5cm]
Иво Алексеев Стратев\\[0.5cm]
Фак. номер: 45342\\[0.2cm]
Група: 3
\end{flushright}
\end{minipage}
\end{center}
\end{titlepage}

\section{E/R диаграма}
\begin{tikzpicture}[node distance=7em]
\node[entity](student){Student};
\node[attribute](number)[above of=student]{\key{number}}edge[line width=1.2pt](student);
\node[attribute](name)[above left of=student]{name}edge[line width=1.2pt](student);
\node[attribute](surname)[above right of=student]{surname}edge[line width=1.2pt](student);
\node[relationship](application)[below of=student]{Application}edge[line width=1.2pt](student);
\node[attribute](session)[above left of=application]{session}edge[line width=1.2pt](application);
\node[attribute](grade)[right of=application]{grade}edge[line width=1.2pt](application);
\node[attribute](date)[above right of=application]{date}edge[line width=1.2pt](application);
\node[entity](exam)[below of=application]{Exam}edge[line width=1.2pt](application);
\node[attribute](ename)[above left of=exam]{name}edge[line width=1.2pt](exam);
\node[attribute](code)[left of=exam]{\key{code}}edge[line width=1.2pt](exam);
\node[attribute](time)[below left of=exam]{time}edge[line width=1.2pt](exam);
\node[attribute](duration)[below of=exam]{duration}edge[line width=1.2pt](exam);
\node[attribute](edate)[below right of=exam]{date}edge[line width=1.2pt](exam);
\node[relationship](offer)[right of=exam]{Offer}edge[line width=1.2pt](exam);
\node[entity](faculty)[right of=offer]{Faculty}edge[<-,line width=1.2pt](offer);
\node[attribute](fname)[above right of=faculty]{name}edge[line width=1.2pt](faculty);
\node[attribute](fcode)[right of=faculty]{\key{code}}edge[line width=1.2pt](faculty);
\node[attribute](address)[below right of=faculty]{address}edge[line width=1.2pt](faculty);
\end{tikzpicture}
\section{Преобразуване на E/R диаграмата до релационни схеми}
\begin{align*}
    Student(\key{number}, name, surname) \\
    Exam(\key{code}, name, date, time, duration, \uwave{faculty}) \\
    Faculty(\key{code}, name, address) \\
    Application(\uwave{\key{number}}, \uwave{\key{code}}, date, session, grade)
\end{align*}
\subsection{Ограничения по референтна цялост}
\begin{align*}
    \pi_{faculty}(Exam) \subseteq \pi_{code}(Faculty) \\
    \pi_{number}(Application) \subseteq \pi_{number}(Student) \\
    \pi_{code}(Application) \subseteq \pi_{code}(Exam)
\end{align*}
\section{Функционални зависимости}
\subsection{За Student}
Има единствена нетривиална функционална зависимост \(number \to name, surname\).
\(number\) е първичния ключ на релацията и следователно релацията \(Student\) е в НФБК, в частност е и в 3НФ.
\subsection{За Exam}
Има единствена нетривиална функционална зависимост \(code \to name, date, time, duration\).
\(date, time, duration\) дори заедно не могат да породат нетривиални ФЗ, понеже в един ден, по едно и също време може да има повече от един изпит с еднаква продължителност.
\(name\) пък не може да породи нетривиална ФЗ, понеже е напълно възможно изпит с едно и също време да се проведе повече от веднъж.
\(code\) е първичния ключ на релацията и следователно релацията \(Exam\) е в НФБК, в частност е и в 3НФ.
\subsection{За Faculty}
Нетривиалните функционални зависимости са:
\begin{align*}
    code \to name, address \\
    name \to code, address
\end{align*}
Понеже и \(\{code\}\) и \(\{name\}\) са кандидат ключове (в частност суперключове), то релацията е в НФБК, в частност е и в 3НФ.
\subsection{За Application}
Единствената нетривиална ФЗ е \(number, code \to date, session, grade\)
и понеже \(\{number, code\}\) е първичен ключ, то \(Application\) е в НФБК, в частност е и в 3НФ.
\section{SQL код за създаване на Базата данни - таблици и ограничения}
\inputminted{sql}{create.sql}
MySQL игнорира ограниченията от тип CHECK.
За това създаваме тригери, които се изпълняват преди добавяне и обновяване
и служат за валидация, че стойността на grade ще е в интервала \([2.0, 6.0]\).
\end{document}